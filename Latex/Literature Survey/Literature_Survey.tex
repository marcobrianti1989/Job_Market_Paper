\documentclass{article}
\usepackage[utf8]{inputenc}
\usepackage[english]{babel}
\usepackage[a4paper,top=3cm,bottom=3cm,left=3cm,right=3cm,%
bindingoffset=0mm]{geometry}
\usepackage{amssymb}
\usepackage{amsmath}
\newtheorem{prop}{Proposition}
\newtheorem{lemma}{Lemma}
\newenvironment{proof}[1][Proof]{\begin{trivlist}
\item[\hskip \labelsep {\bfseries #1}]}{\end{trivlist}}
\newcommand{\qed}{\nobreak \ifvmode \relax \else
      \ifdim\lastskip<1.5em \hskip-\lastskip
      \hskip1.5em plus0em minus0.5em \fi \nobreak
      \vrule height0.75em width0.75em depth0em\fi}
\usepackage{tikz}
\usepackage{graphicx}
\usepackage{rotating}
\usepackage{float}
\linespread{1.3}
\raggedbottom




%
\font\reali=msbm10 at 12pt
% subsets of real numbers
\newcommand{\real}{\hbox{\reali R}}
\newcommand{\realp}{\hbox{\reali R}_{\scriptscriptstyle +}}
\newcommand{\realpp}{\hbox{\reali R}_{\scriptscriptstyle ++}}
\newcommand{\R}{\mathbb{R}}
\DeclareMathOperator{\E}{\mathbb{E}}
%

\title{Related Literature}
\author{Marco Brianti}
\date{A.Y. 2018/2019}

\begin{document}
	\large{

\maketitle

\section*{Angeletos and La'O (2010) - NBER Macroeconomics Annual}

They introduce heterogeneous information in a Real Business Cycle model. This assumption can have profound implications for the business cycle. They analyze a standard RBC model with no capital augmented with dispersed information frictions. In particular, economic decisions have to be made under heterogeneous information about the aggregate shocks hitting the economy. They summarize their results as follows. (i) Dispersed information induces inertia in the response of macroeconomic outcomes. (ii) Dispersion of information induces technology shocks to explain only a small fraction of the high-frequency variation in the business cycle. (iii) The drivers of the residual variation in the short-run fluctuations is simply the noise in available information, i.e. correlated errors in the agents' expectations of the fundamental shocks. (iv) These noise-driven fluctuations help formalize a certain type of demand shocks. (v) If a social planner takes information dispersion as given then equilibrium is already efficient implying no room for any intervention.

Importantly, what drivers their results is not per se the level of uncertainty about the underlying fundamental but rather the lack of common knowledge about it. Indeed, their effects are consistent with an arbitrary small level of uncertainty about the underlying fundamentals. However, at the same time, the lack of common knowledge alone does not explain the magnitude of our effects. What they need is also a degree of strategic complementary, which means that dispersed information has an effect if agents care on other agents' choices. Their findings hinge on the combination of heterogeneous information with strong strategic complementarity - but they do not hinge on the level of uncertainty about underlying fundamentals. In addition, notice that standard noise-shock literature obtains fluctuations that vanish when uncertainty on fundamentals is small enough.

Fluctuations are not generated by uncertainty regarding future exogenous fundamentals but by uncertainty regarding future choices of other agents that have different information. Specifically, when information is asymmetric agents face additional uncertainty about the level of economic activity beyond the one they face about fundamentals. It is specifically this feature that differentiated dispersed information different from uncertainty in fundamentals. Conversely, strategic complementary is irrelevant for business cycle fluctuations when information is commonly shared. Interestingly, the larger the level of strategic complementarity the less agents focus on fundamental shocks and the more they focus on public signals attempting to coordinate with each other. Thus, it follows that stronger strategic complementarity induces equilibrium to be more anchored to the past aggregate fundamentals, more sensitive to public information and less sensitive to private information.

Another important point that they stress is that the variance of the idiosyncratic signal received by agents is different from the degree of information dispersion. For example, if the variance of the idiosyncratic noise rises, agents might decide to rely less on their private signal focusing more on the public signal converging expectations and thus decreasing information dispersion. 



}
\end{document}

