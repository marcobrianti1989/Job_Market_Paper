\documentclass[hyperref={pdfpagelabels=false}]{beamer}
\usepackage{graphicx,lmodern,subfigure,ulem,color,graphicx,tikz,booktabs,natbib}
\usepackage{mathrsfs}
\usetheme{Warsaw}
%\definecolor{beamer@blendedblue}{rgb}{0.1,0.5,0.1}
%\definecolor{ForestGreen}{RGB}{60, 140, 60}
%\setbeamercolor{structure}{fg=beamer@blendedblue}
\setbeamertemplate{navigation symbols}{}
\setbeamertemplate{footline}[frame number]
\bibliographystyle{chicago}
\newcommand{\spitem}{\vspace{.3cm}\item}
\newcommand{\elas}{$E_{labor}$}
\def \ourFigPath {../../} 


\title{Noise Shocks and Heterogeneous Agents}
\author{Marco Brianti}
\institute{Boston College}
\date{2017}


\begin{document}
	
	\frame{\titlepage \begin{center} Dissertation Workshop \end{center} }
	
	\frame{\frametitle{Research Question (Preliminary)}
		
		The key empirical question is how \textbf{information dispersion} across agents affect economic fluctuations.
		
		\
		
		\
		
		\
		
		\
		
		Two possible analysis
		
		\
		
		\begin{enumerate}
			\item Identifying an \textbf{information-dispersion shock}
			
			\
			\item Analyzing the effect of fundamental shocks conditioning on different level of information dispersion
		\end{enumerate}
		
		
		
	}
	
	
	
	\frame{\frametitle{Background (I)}
		
		A standard \textbf{noise shock} is defined as a noisy public signal regarding aggregate future fundamentals.
		
		\
		
		\
		
		\
		
		In this case agents \textbf{coordinate} their choices with respect to an aggregate biased signal.
		
		\
		
		\
		
		\
		
		Underlying assumption is that the \textbf{quality} of the signal is low.
		
	}
	
	
	\frame{\frametitle{Background (II)}
		
		An \textbf{information-dispersion shock} is defined as a signal which spread out expectations regarding aggregate future fundamentals.
		
		\
		
		\
		
		\
		
		In this case agents \textbf{fail to coordinate} their choices with respect to expected future fundamentals.
		
		\
		
		\
		
		\
		
		Underlying assumption is that the \textbf{quantity} of the signal is low.
		
	}
	
	
	\frame{\frametitle{Formalization}
		
		Consider a simple economy populated by $I$ agents where each agent $i$ attempt to forecast fundamental variable $x_{t+1}$ given the information set at time $t$.
		
		\
		
		Define the mean squared forecast error across agents as follows,
		
\begin{eqnarray}\label{eq:FE_shock}
\phi_t = I^{-1} \sum_{i=1}^I \big\{ E_t^i [ x_{t+1} ] - x_{t+1} \big\}^2 
\end{eqnarray}


where

\begin{itemize}
	\item $E^i_t [ x_{t+1} ]$ is the expectation of agent $i$ on $x_{t+1}$
	\item $E^i_t [ x_{t+1} ] - x_{t+1}$ is the forecast error of agent $i$ on $x_{t+1}$
\end{itemize}   

\

\textbf{Intuition.} $\phi_t$ represents the precision of agents' expectations in the whole economy.
				
	}



\frame{\frametitle{Decomposition}
	
	
	Now, the interesting part of Equation \ref{eq:FE_shock} is that can be decomposed as follows,
	\begin{eqnarray}
	\begin{aligned}
	\phi_t &= I^{-1} \sum_{i=1}^I \big\{ E_t^i ( x_{t+1} ) - x_{t+1} \big\}^2 \\
	&= I^{-1} \sum_{i=1}^I \big\{  E_t^i ( x_{t+1} )^2 - 2E_t^i ( x_{t+1} )x_{t+1} + x_{t+1}^2  \big\} \\
	&= I^{-1} \sum_{i=1}^I E_t^i ( x_{t+1} )^2 - 2\bar{x}_{t,t+1}^i x_{t+1} + x_{t+1}^2 \\
	&= I^{-1} \sum_{i=1}^I E_t^i ( x_{t+1} )^2 - (\bar{x}_{t,t+1}^i)^2 + (\bar{x}_{t,t+1}^i)^2 - 2\bar{x}_{t,t+1}^i x_{t+1} + x_{t+1}^2 \\
	&= Var^i_t(x_{t+1}) + (\bar{x}_{t,t+1}^i - x_{t+1})^2, \\
	\end{aligned}
	\end{eqnarray}
	
where $\bar{x}_{t,t+1}^i = I^{-1} \sum_{i=1}^I E_t^i ( x_{t+1} )$ is the average expectation across agents of $x_{t+1}$ given the information set at time $t$.
	
	
}

\frame{\frametitle{Intuition}
	
	From previous slide,
	\begin{eqnarray}\label{eq:decomposition}
	\phi_t = Var^i_t(x_{t+1}) + (\bar{x}_{t,t+1}^i - x_{t+1})^2
	\end{eqnarray}
	
	\
	
	
	Equation \ref{eq:decomposition} is divided into two parts: 
	
	\
	
	
	\begin{enumerate}
		\item $Var^i_t(x_{t+1})$, which is the variance across agents of expectations of $x_{t+1}$ given information set at time $t$,
		
		\
		
		\item $(\bar{x}_{t,t+1}^i - x_{t+1})^2$, which is the square of difference between the average expectation across agents of $x_{t+1}$ at time $t$ and its actual realization.
	\end{enumerate}

}

\frame{\frametitle{Econometrics}
	
	
An \textbf{information-dispersion shock} can be identified as
\begin{eqnarray}\label{eq:idio_shock}
\iota_t = Var^i_t(x_{t+1})
\end{eqnarray}
and a standard \textbf{noise shock} as
\begin{eqnarray}\label{eq:aggre_shock}
\eta_t = \bar{x}_{t,t+1}^i - x_{t+1}
\end{eqnarray}
Thus, $\phi_t$ can be represented as the sum of two shocks
\begin{eqnarray}\label{eq:bias_var_dec}
\phi_t = \iota_t + \eta_t^2
\end{eqnarray}
which is simply the Variance-Bias decomposition.

\

\textbf{Intuition.} \textbf{Information-dispersion shocks} can be interpreted as situations where aggregate information is weak across agents while \textbf{noise shocks} as situation where aggregate information is biased.




}
	

\end{document}