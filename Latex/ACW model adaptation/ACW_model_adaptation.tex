\documentclass[hyperref={pdfpagelabels=false}]{beamer}
\usepackage{graphicx,lmodern,subfigure,ulem,color,graphicx,tikz,booktabs,natbib}
\usepackage{mathrsfs}
\usetheme{Warsaw}
%\definecolor{beamer@blendedblue}{rgb}{0.1,0.5,0.1}
%\definecolor{ForestGreen}{RGB}{60, 140, 60}
%\setbeamercolor{structure}{fg=beamer@blendedblue}
\setbeamertemplate{navigation symbols}{}
\setbeamertemplate{footline}[frame number]
\bibliographystyle{chicago}
\newcommand{\spitem}{\vspace{.3cm}\item}
\newcommand{\elas}{$E_{labor}$}
%\def \FigPath {Users\th3\Documents\Job_Market_Paper\Code\Figures} 

\font\reali=msbm10 at 12pt
% subsets of real numbers
\newcommand{\real}{\hbox{\reali R}}
\newcommand{\realp}{\hbox{\reali R}_{\scriptscriptstyle +}}
\newcommand{\realpp}{\hbox{\reali R}_{\scriptscriptstyle ++}}
\newcommand{\R}{\mathbb{R}}
\DeclareMathOperator{\E}{\mathbb{E}}


\title{Three-Period Model}
\author{Marco Brianti}
\institute{Boston College}


\begin{document}
	
	\frame{\titlepage \begin{center} Dissertation Project \end{center} }
	

\frame{\frametitle{Introduction to the Model}

To provide an economic intuition of the differential response of \textbf{cash holdings} to uncertainty and financial shocks, I present a properly augmented model in the spirit of 
\begin{itemize}
	\item Almeida, Campello, and Weisbach (2004)
	\item Han and Qiu (2007)
\end{itemize}

\

\

\

It is a simple representation of a dynamic setting where a profit-maximizing firm have
\begin{itemize}
	\item present and future investment opportunities
	\item current cash flow and external sources of finance might not be enough to fund all profitable projects
\end{itemize}


}


\frame{\frametitle{Main Features}
	
	\begin{itemize}
	
	\item Model has three periods: 0, 1 and 2.
	
	\
	
	\item There is one representative firm (or a continuum of it)
	
	\
	
	\item Discount factor $\beta = 1$, but it can easily be relaxed
	
	\
	
	\item In P0 firm can invest $I_0$ in a long-term project.
	\begin{itemize}
		\item $I_0$ pays a deterministic return $g(I_0) = G(I_0) + qI_0$ in P2
	\end{itemize}
	
	\

    \item In P1 firm can invest $I_1$ in a short-term project.
    \begin{itemize}
	\item $I_1$ pays a deterministic return $h(i_0) = H(i_0) + qI_0$ in P2
    \end{itemize}

    \
    
    \item Both $G(\cdot)$ and $H(\cdot)$ - and thus $g(\cdot)$ and $h(\cdot)$ - display the following properties
    \begin{itemize}
    	\item $G'(\cdot)$ and $H'(\cdot)$ strictly positive
    	\item $G''(\cdot)$ and $H''(\cdot)$ strictly negative
    	\item $G'''(\cdot)$ and $H'''(\cdot)$ strictly positive
    \end{itemize}

   \end{itemize}	
	
}

\frame{\frametitle{Period 0}
	
	
	\begin{itemize}
		\item Firm enters the period with $c_0$ internal liquidity from past and current cash flows
		
		\item Firm chooses optimal level of investment $I_0$, cash holding $C$, and borrowing $B_0$

			\item Optimal choices are subject to nonnegative dividends constraint,
			$$
			d_0 = c_0 + B_0 - I_0 - C \geq 0
			$$
            and borrowing constraint 
			$$
			0 \leq B_0 \leq (1 - \tau)qI_0
$$			
where 

\begin{itemize}
	\item $q \in (0,1]$ is the part of $I_0$ that can be liquidated after usage 
	\item $1 - \tau$ is the part of the liquidation value of $I_0$ that can be captured by creditors
\end{itemize}
			
\end{itemize}

}


\frame{\frametitle{Period 1}

\begin{itemize}
	\item Firm enters the period with $C + c_1$ internal liquidity where
	\begin{itemize}
		\item $C$ is optimal level of cash holding chosen in P0
		\item $c_1 \sim F[\underline{c_1}, \  \overline{c_1}] \geq 0$ is current cash flow
		\item $c_1$ is unknown in P0 and drawn at the beginning of P1 
		\end{itemize}
	
	\
	
	\item Firm chooses optimal schedules of both investment $I_1(c_1)$ and borrowing $B_1(c_1)$
	
	\
	
	\item Optimal choices are subject to nonnegative dividends constraint,
	$$
	d_1 = c_1 + B_1(c_1) - I_1(c_1) + C \geq 0
	$$
	and borrowing constraint 
	$$
	0 \leq B_1(c_1) \leq (1 - \tau_1)qI_1(c_1)
	$$
\end{itemize}

}


\frame{\frametitle{Period 2}
	
	\begin{itemize}
		
		\item Firm receives deterministic returns $g(I_0)$ and $h(I_1(c_1))$
		
		\
		
		\item Firm pays back loans $B_0$ and $B_1(c_1)$
		
		\
		
		\item Dividends are defined as
		$$
		d_2 = g(I_0) + h(I_1(c_1)) - B_0 - B_1(c_1)
		$$
				
	\end{itemize}
		
}

\frame{\frametitle{Firm's Problem}
	
\begin{eqnarray}
\max_{C,I_0,B_0,I_1(c_1),B_1(c_1)} d_0 + d_1 + d_2
\end{eqnarray}	
subject to
\begin{eqnarray}
\begin{aligned}
d_0 &= c_0 + B_0 - I_0 - C \geq 0 \\
d_1 &= c_1 + B_1(c_1) - I_1(c_1) + C \geq 0 \\
d_2 &= g(I_0) + h(I_1(c_1)) - B_0 - B_1(c_1) \\
0   &\leq B_0 \leq (1 - \tau)qI_0 \\
0   &\leq B_1(c_1) \leq (1 - \tau)qI_1(c_1)
\end{aligned}
\end{eqnarray}	
}


\frame{\frametitle{Solution - Unconstrained Firms}
	
A firm is financially unconstrained if it has enough financial resources such that

$$
g'(I_0^*) = 1
$$	

and

$$
h'(I_1^*(c_1)) = h'(I_1^*) = 1 \ \ \ \forall \ c_1 \in [\underline{c_1}, \  \overline{c_1}]
$$

\

\begin{itemize}
\item Both $I_0^*$ and $I_1^*$ are independent of $c_1$
\item Firm is indifferent on optimal $C^*$, $B_0^*$, and $B_1^*$
\end{itemize}
		
}


\frame{\frametitle{Solution - Constrained Firms (I)}

A firm is financially constrained if its investment levels are always lower than the first-best levels $I_0^*$ and $I_1^*$.

\

\

\


Constraints are always expected to bind since it is not profitable to 

\

\begin{itemize}
	\item paying out dividends in the first two periods
	
	\
	
	\item borrowing less than the maximum amount
\end{itemize}  





}

\frame{\frametitle{Solution - Constrained Firms (II)}

Since constraints bind, 
$$
I_0 = \frac{c_0 - C}{1 - q + q \tau } \ \ \text{and} \ \ I_1(c_1) = \frac{c_1 + C}{1 - q + q \tau }
$$

\

\

Firm faces now a trade of on choosing optimal cash holding $C^*$

\


\begin{itemize}
	\item the cost of saving an additional unit of cash holding $C$ is forgoing a unit of current investment projects
	
	\
	
	
	\item the benefit of saving an additional unit of cash holding $C$ is the higher ability to fund future investment projects
\end{itemize}
	
}


\frame{\frametitle{Solution - Constrained Firms (III)}
	
Optimal $C^*(0,F)$ should be chosen to maximize
\begin{eqnarray}
\begin{aligned}
\text{Objective} \ \ &= \ \ g(I_0) - I_0 + \E \big[ h(I_1) - I_1 | F \big] \\
&= \ \ g \bigg( \frac{c_0 - C}{1 - q + q \tau } \bigg) - \frac{c_0 - C}{1 - q + q \tau } \\
&+ \ \  \E \bigg[  h \bigg( \frac{c_1 + C}{1 - q + q \tau } \bigg) - \frac{c_1 + C}{1 - q + q \tau }   \bigg| F     \bigg]
\end{aligned}
\end{eqnarray}	

\

Solution for $C^*(0,F)$ solves
\begin{eqnarray}
g' \bigg( \frac{c_0 - C^*(0,F)}{1 - q + q \tau } \bigg) = \E \bigg[  h' \bigg( \frac{c_1 + C^*(0,F)}{1 - q + q \tau } \bigg)  \bigg| F  \bigg]
\end{eqnarray}	
	where SOC is negative by assumption on $G(\cdot)$ and $H(\cdot)$.
	
}

\frame{\frametitle{Analysis}
	
	I analyze the effect of an uncertainty and credit supply shocks in period 0 in order to see the differential response of cash holding $C^*$
	
	\
	
	\
	
	
	\textbf{Uncertainty Shock.} Mean-preserving spread of perceived distribution $F$ of $c_1$.
	
	\
	
	\
	
	\textbf{Credit Supply Shock.} Decrease by $\varepsilon$ of the part of the liquidation value of $I_0$ that can be captured by creditors,
	$$
	B_0 \leq (1 - (\tau + \varepsilon))qI_0
	$$
	
}


\frame{\frametitle{Uncertainty Shock (I)}
	
	Assume that $C^*(0,F)$ is the optimal cash holding when distribution of $c_1$ is $F$ and no credit supply shocks are in place.
	
	\
	
	Then the following two relations hold,
	
	
	\begin{eqnarray}\label{eq:optimal_sol_F}
	g' \bigg( \frac{c_0 - C^*(0,F)}{1 - q + q \tau } \bigg) = \E \bigg[  h' \bigg( \frac{c_1 + C^*(0,F)}{1 - q + q \tau } \bigg)  \bigg| F  \bigg]
	\end{eqnarray}	
	
	
	and
	
	
	\begin{eqnarray}\label{eq:jensen_ineq}
    \E \bigg[  h' \bigg( \frac{c_1 + C^*(0,F)}{1 - q + q \tau } \bigg)  \bigg| Q  \bigg] > \E \bigg[  h' \bigg( \frac{c_1 + C^*(0,F)}{1 - q + q \tau } \bigg)  \bigg| F  \bigg]
    \end{eqnarray}
    
    where 
    
    \begin{itemize}
    	\item Equation \ref{eq:optimal_sol_F} holds because $C^*(0,F)$ is optimal cash flow with $F$ 
    	\item Equation \ref{eq:jensen_ineq} holds because of Jensen's inequality. 
    \end{itemize}    	

		
}

\frame{\frametitle{Uncertainty Shock (II)}
	
Combining \ref{eq:optimal_sol_F} and \ref{eq:jensen_ineq} yields


\begin{eqnarray}
g' \bigg( \frac{c_0 - C^*(0,F)}{1 - q + q (\tau + \varepsilon) } \bigg) > \E \bigg[  h' \bigg( \frac{c_1 + C^*(0,F)}{1 - q + q \tau } \bigg)  \bigg| F  \bigg]
\end{eqnarray}

\	

	which implies that $C^*(0,Q) > C^*(0,F)$.	

\

\

\textbf{Result.} In response to an uncertainty shock firm tends to accumulate additional cash holding from the motive of precautionary savings.

	
}

\frame{\frametitle{Credit Supply Shock (I)}%TO BE FINISHED
	
	Given $C^*(0,F)$, then the following two relations hold,
	
	\begin{eqnarray}\label{eq:optimal_sol_0}
	g' \bigg( \frac{c_0 - C^*(0,F)}{1 - q + q \tau } \bigg) = \E \bigg[  h' \bigg( \frac{c_1 + C^*(0,F)}{1 - q + q \tau } \bigg)  \bigg| F  \bigg]
	\end{eqnarray}	
	
	and
	
	\begin{eqnarray}\label{eq:jensen_ineqeps}
	g' \bigg( \frac{c_0 - C^*(0,F)}{1 - q + q (\tau + \varepsilon)} \bigg) > g' \bigg( \frac{c_0 - C^*(0,F)}{1 - q + q \tau } \bigg)
	\end{eqnarray}
	
	where 
	
	\begin{itemize}
		\item Equation \ref{eq:optimal_sol_0} holds because $C^*(0,F)$ is optimal cash flow with $F$ 
		\item Equation \ref{eq:jensen_ineqeps} holds because of Jensen's inequality. 
	\end{itemize}    	
}

\frame{\frametitle{Credit Supply Shock (II)}
	
	Combining \ref{eq:optimal_sol_0} and \ref{eq:jensen_ineqeps} yields
	
	
	\begin{eqnarray}
	g' \bigg( \frac{c_0 - C^*(0,F)}{1 - q + q \tau } \bigg) < \E \bigg[  h' \bigg( \frac{c_1 + C^*(0,F)}{1 - q + q \tau } \bigg)  \bigg| Q  \bigg]
	\end{eqnarray}
	
	\	
	
	which implies that $C^*(\varepsilon,F) < C^*(0,F)$.	
	
	\
	
	\
	
	\textbf{Result.} In response to a negative credit supply shock firm tends to accumulate less cash to finance current projects which are more financially constrained. 

}

\frame{\frametitle{Lesson learned}
	
As a precautionary motive, firms prefer to increase cash holdings if they expect future outcomes to be more volatile.

\

After a financial shock, firms substitute more cash today with less cash tomorrow because they expect to me more financially constrained today than tomorrow. 	
	
}

\end{document}