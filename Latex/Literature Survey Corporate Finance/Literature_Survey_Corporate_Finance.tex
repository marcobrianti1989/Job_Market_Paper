\documentclass{article}
\usepackage[utf8]{inputenc}
\usepackage[english]{babel}
\usepackage[a4paper,top=3cm,bottom=3cm,left=3cm,right=3cm,%
bindingoffset=0mm]{geometry}
\usepackage{amssymb}
\usepackage{amsmath}
\newtheorem{prop}{Proposition}
\newtheorem{lemma}{Lemma}
\newenvironment{proof}[1][Proof]{\begin{trivlist}
\item[\hskip \labelsep {\bfseries #1}]}{\end{trivlist}}
\newcommand{\qed}{\nobreak \ifvmode \relax \else
      \ifdim\lastskip<1.5em \hskip-\lastskip
      \hskip1.5em plus0em minus0.5em \fi \nobreak
      \vrule height0.75em width0.75em depth0em\fi}
\usepackage{tikz}
\usepackage{graphicx}
\usepackage{rotating}
\usepackage{float}
\linespread{1.3}
\raggedbottom




%
\font\reali=msbm10 at 12pt
% subsets of real numbers
\newcommand{\real}{\hbox{\reali R}}
\newcommand{\realp}{\hbox{\reali R}_{\scriptscriptstyle +}}
\newcommand{\realpp}{\hbox{\reali R}_{\scriptscriptstyle ++}}
\newcommand{\R}{\mathbb{R}}
\DeclareMathOperator{\E}{\mathbb{E}}
%

\title{Related Literature}
\author{Marco Brianti}
\date{A.Y. 2018/2019}

\begin{document}
	\large{

\maketitle

\tableofcontents

\section{Almeida, Campello, and Weisbach (2004) - JF}

Two important areas of research in corporate finance are the effects of financial constraints on firm behavior and the manner in which firms perform financial management. These two issues, although often studied separately, are fundamentally linked.  As originally proposed by Keynes (1936), a major advantage of a liquid balance sheet is that it allows firms to undertake valuable projects when they arise. However, Keynes also argued that the importance of balance sheet liquidity is influenced by the extent to which firms have access to external capital market. If a firm has unrestricted access to external capital (financially unconstrained) there is no need to safeguard against future investment needs and corporate liquidity becomes irrelevant. In contrast, when the firm faces financing frictions, liquidity management may become a key issues for corporate policy. Firms anticipating financing constraints in the future respond to those potential constraints by hoarding cash today. Holding cash is costly, nonetheless, since higher cash savings require reductions in current, valuable investments. Constrained firms thus choose their optimal cash policy to balance the profitability of current and future investments.

Their basic model is a simple representation of a dynamic problem in which the firm has both present and future investment opportunities, and in which cash flows from assets in place might not be sufficient to fund all positive projects. Depending on the firm's capacity for external finance, hoarding cash may facilitate future investment. The model has three dates and firm has investment opportunity in both second and last period. Firm uses cash and borrowing to finance both investments. In this setup, firm is concerned only about whether or not to store cash in the first or second period to finance both investment opportunities. Financial constraints arise from the fact that only a part of future flows can be pledged as collateral and banks do not lend more than this partial amount. The crucial feature is that some firms have some limitations in their capacity to raise external finance, and that such limitations may cause those firms to invest below first-best levels. For a financially constrained firms, holding cash entails both costs and benefits. A constrained firm cannot undertake all of its positive projects, so holding cash is costly because it requires sacrificing some valuable investment projects today. The benefit of cash is the increase in the firm's ability to finance future projects that might become available. Optimal cash policies arise as a trade-off between these costs and benefits, both of which are generated by the same underlying capital market imperfection. \textbf{Proposition.} The cash flow sensitivity of cash (how much increase cash holding for an additional dollar of cash flow available) has the following properties: (i) positive for financially constrained firms; (ii) indeterminate for financially unconstrained firms.

They then test the model's main prediction using firm level data from COMPUSTAT. Their simplest test regresses current cash holding-to-assets changes on current cash flow-to-assets and Tobin's Q and size control. They then separate financially constrained firms over non-financially constrained using 5 different approaches as robustness check. According to descriptive analysis, constrained firms hold far more cash on their balance sheet. Moreover, the set of constrained firms displays significantly positive sensitivities of cash to cash flow, while unconstrained firms show insignificant cash-cash flow sensitivities. Thus, there are systematic differences between constrained and unconstrained firms in the way they conduct their cash policies, and that these differences are manifested along the lines suggested by their theory. Finally, they also show that during a downturn, constrained firms tend to stare relatively more cash from cash flow. This should happen because these periods are characterized both by an increase in the marginal attractiveness of future investments (when compared to current ones), as well as by a decline in current income flows.




}

\end{document}

