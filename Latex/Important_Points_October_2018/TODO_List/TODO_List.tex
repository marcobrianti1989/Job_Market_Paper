\documentclass{article}
\usepackage[utf8]{inputenc}
\usepackage[english]{babel}
\usepackage[a4paper,top=3cm,bottom=3cm,left=3cm,right=3cm,%
bindingoffset=0mm]{geometry}
\usepackage{amssymb}
\usepackage{amsmath}
\newtheorem{prop}{Proposition}
\newtheorem{lemma}{Lemma}
\newenvironment{proof}[1][Proof]{\begin{trivlist}
\item[\hskip \labelsep {\bfseries #1}]}{\end{trivlist}}
\newcommand{\qed}{\nobreak \ifvmode \relax \else
      \ifdim\lastskip<1.5em \hskip-\lastskip
      \hskip1.5em plus0em minus0.5em \fi \nobreak
      \vrule height0.75em width0.75em depth0em\fi}
\usepackage{tikz}
\usepackage{graphicx}
\usepackage{rotating}
\usepackage{float}
\linespread{1.3}
\raggedbottom




%
\font\reali=msbm10 at 12pt
% subsets of real numbers
\newcommand{\real}{\hbox{\reali R}}
\newcommand{\realp}{\hbox{\reali R}_{\scriptscriptstyle +}}
\newcommand{\realpp}{\hbox{\reali R}_{\scriptscriptstyle ++}}
\newcommand{\R}{\mathbb{R}}
\DeclareMathOperator{\E}{\mathbb{E}}
%

\title{To-Do List}
\author{Marco Brianti}
\date{A.Y. 2018/2019}

\begin{document}
	\large{

\maketitle

\section*{10/11/2018}

\begin{itemize}
	\item You need to control for zero-lower bound and quantitative easing.
	\item Opler, Pinkowitz, Stulz, and Williamson (1999) and Bates, Kahle, and Stulz (2009) discuss empirical facts of cash holdings in United States. If you use cash holdings as an instrument you need to read both papers.
	\item Duchin, Ozbas, and Sensoy (2010) support presence of credit shocks during the financial crisis and show that firms with more cash holdings cut investment by less. You need to read it. 
	\item You need to read Christiano, Motto, and Rostagno (2013). Paper regarding the effect of risk shocks.
	\item Find a consistent definition of an unanticipated financial shock.
	\item Harford, Klasa, Maxwell (2013) is potentially an interesting paper. They find that firms mitigate refinancing risks by increasing their cash holdings by saving from cash flow. It might be useful to discuss propagation effects. Uncertainty affects much more the risk of being financially constrained in future rather than being financially constrained today! Key point I have to make! Uncertainty remember more expected financial shocks.
	\item Also Duchin (2010) JF is a potentially interesting paper. He shows that multiproduct firms hold less cash because better diversified and thus facing less uncertainty.
\end{itemize}


}
\end{document}

